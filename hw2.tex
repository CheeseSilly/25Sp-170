% search for all TODOs

\documentclass[11pt]{article}
\usepackage{cs170}


\def\title{Homework 2}
\def\duedate{2025/2/23, at 10:00 am (grace period until 11:59pm)}

\begin{document}
\maketitle
Due \textbf{\duedate}


\question{Study Group}
List the names and SIDs of the members in your study group.
If you have no collaborators, you must explicitly write ``none''.

\begin{solution} I worked on this homework with the following collaborators:
\begin{itemize}
    \item none,which is only me,Sillycheese
\end{itemize}
\end{solution}

\question{Median of Medians,The Resistance,Poker}

\begin{subparts}
    \item (2 points) Let us see an example of QuickSelect in action. Suppose you always pick
    the first element as the pivot. Compute QuickSelect(A, 6) for the following array: \par
    \begin{solution}
        [78,13,97,45,48,26,85,100,78] k=4 \par
        [13,45,48,26] k=4 \par
        [45,48,26] k=3 \par
        [48] k=1 \par
    \end{solution}

    \item (2 points) Consider the array \par
    shuffled into some arbitrary order. What is the worst-case runtime of QuickSelect(A,
n/2) in terms of n? Construct a sequence of ‘bad’ pivot choices that achieves this worst-
case runtime. \par
    \begin{solution}
        $$\frac{n(n+1)}{2}=O(n^2)$$
    \end{solution}

    \item (3 points) Let p be the pivot chosen by DeterministicSelect on A. Show that at
    least 3n/10 elements in A are less than or equal to p, and that at least 3n/10 elements
    are greater than or equal p. \par
    \begin{solution}
        firstly,at least half of medians array is less than p.so it is $n/10$.
        then go back to every median's subarray and add them.It is $\frac{2n+n}{10}$(also need to add itself).
        so at least $\frac{3n}{10}$ elements in A are Less than or equal to p. \par
        Same to the greater issue.
    \end{solution}

    \item In this problem, we will show that the worst-case runtime of DeterministicSelect(A,
    k) using the ‘Median of Medians’ strategy is O(n). \par
    \begin{enumerate}[i.]
        \item Find a recurrence relation for the time complexity of the algorithm, T (n). \par
        \begin{solution}
            $$T(n)<=T(n/5)+T(7n/10)+O(n)$$ \par
            $T(n/5)$ is to find the medians in each subarray.\par
            $T(7n/10)$ is the at most size of partition size,which is used to recursive call to the DC \par
            $O(n)$ is just build array and partition time.
        \end{solution}

        \item Use the recurrence relation to show that, for some sufficently large c ≥ 0, the
        inequality T (n) ≤c ·n always holds. \par
        \begin{solution}
            $$T(n)\leq c(n/5)+c(7n/10)+O(n) \leq c$$
        \end{solution}
    \end{enumerate}

    \item We are playing a variant of The Resistance, a board game where there are n players, s of which are spies. In this variant, in every round, we choose a subset of players to go on a mission. A mission succeeds if the subset of the players does not contain a spy, but fails if at least one spy goes on the mission. After a mission completes, we only know its outcome and not which of the players on the mission were spies. \par
    \begin{enumerate}[(a)]
        \item (2 points) If there is one spy (s = 1), come up with a strategy that identifies the spy in O(log(n)) missions. You do not need to prove that your strategy works. \par
        \begin{solution}
            partition n/2 each time.until we found that spy. it looks like a BS algorithm. \par
        \end{solution}
        
        \item (2 points) In the general case, when there are s spies, consider evenly splitting the n players into x disjoint groups (containing ≈ n/x players each), and send each group on a mission. At least how many of these x missions must succeed, in terms of x and s? \par
        \begin{solution}
            x-s
        \end{solution}

        \item (8 points) Come up with a strategy that identifies all the spies in O(s log(n/s)) missions. \par
        \begin{solution}
            we can partition 2s groups.at least s groups will succeed.then we can recursively deal with those which failed.
            so the $O(n)=s\log(n/s)$.\par
            we need to find s groups each time,and we also need to deal with recursive tree,which is $O(log(n/s))$
        \end{solution}
    \end{enumerate}

    \item You are playing poker with n other friends. Some of your friends are “truthers,” meaning that they always tell the truth, while everyone else is a “bluffer,” meaning that they sometimes tell the truth but also sometimes bluff (lie). You do not know who is a truther or a bluffer, but all your friends do. All you know is that there are more people who always tell the truth than people who bluff. Your goal is to identify one specific player that is a truther. You are allowed to perform a ‘query’ operation as follows: you pick two players as partners. You ask each player if their partner is a truther or bluffer. When you do this, truthers will always tell the true identity of their partner, but bluffers can either lie or tell the truth. Your strategy should work regardless of whether bluffers lie or tell the truth. \par
    \begin{enumerate}[(a)]
            \item(3 points) For a given player x, devise a strategy that returns whether or not x is a truther using O(n) queries. Just an informal description of your test and a brief explanation of why it works is needed.\par
            \begin{solution}
                ask others(n-1) to determine whether x is truther.
            \end{solution}

            \item(10 points) Devise a divide and conquer algorithm that finds a truther in O(n log n) queries (where one query is taking two players x and y and asking x to identify y and y to identify x). Hint: Split the players into two groups, recurse on each group, and use part (a). What property must hold for at least one of the two groups? \par
            \begin{solution}
                split the players into 2 groups ,and recurse on each group using part(a).  \par
                By doing this,the number of queries that your strategy requires in the worst case is $2T(\frac{n}{2})+O(n)$
            \end{solution}
    \end{enumerate}

\end{subparts}

\end{document}
